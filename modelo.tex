\documentclass[11pt,portuguese]%{article}
{scrartcl}
\usepackage[utf8]{inputenc}
\usepackage[portuguese]{babel}
\usepackage[T1]{fontenc}
\usepackage{amsmath}
\usepackage{amsthm}
\usepackage{amsfonts}
\usepackage{amssymb}
\usepackage{lmodern}
\usepackage{fourier}
\usepackage{graphicx}
\usepackage[left=3cm,right=3cm,top=3cm,bottom=3cm]{geometry}

%----------------------------------------------------------------------------------------
%	TIKZ 
%-------------------------------------------------------------------------------------
\usepackage{tikz} % for diagrams
\usepackage{tikz-3dplot}
\usetikzlibrary{quotes,angles,patterns,external,fit,intersections,matrix,positioning,fadings,arrows,calc,decorations.pathmorphing,decorations.pathreplacing,shapes}

\usepackage{tikz-cd}

\tikzset{snake it/.style={-stealth,
decoration={snake, 
    amplitude = .4mm,
    segment length = 2mm,
    post length=0.9mm},decorate}}
    \usepackage{pgfplots}

%----------------------------------------------------------------------------------------
%	AMBIENTES 
%-------------------------------------------------------------------------------------
\newtheorem{teo}{Teorema}[section]
\newtheorem{prop}[teo]{Proposição}
\newtheorem{conjecture}[teo]{Conjectura}
\newtheorem{coro}{Corolário}[teo]
\newtheorem{defi}{Definição}[section]
\theoremstyle{defi}
\newcommand{\bdefi}{\begin{defi}}
\newcommand{\edefi}{\end{defi}}
\newcommand{\bprop}{\begin{prop}}
\newcommand{\eprop}{\end{prop}}
\renewcommand\qedsymbol{$\blacksquare$}
\newtheorem{ex}{Exemplo}[section]
\theoremstyle{ex}
\newcommand{\bex}{\begin{ex}}
\newcommand{\eex}{\end{ex}}

\newcommand{\mais}{{\textsc{\footnotesize +}}}
\newcommand{\menos}{{\textsc{\footnotesize $-$}}}

%----------------------------------------------------------------------------------------
%	MEUS COMANDOS 
%-------------------------------------------------------------------------------------
\newcommand{\id}{\mathbbmtt{1}}
\newcommand{\ee}{\textbf{e}}
\newcommand{\vv}{\textbf{v}}
\newcommand{\uu}{\textbf{u}}
\newcommand{\ww}{\textbf{w}}
\newcommand{\nn}{\textbf{n}}
\newcommand{\pp}{\textbf{p}}
\newcommand{\xx}{\textbf{x}}
\newcommand{\yy}{\textbf{y}}
\newcommand{\zz}{\textbf{z}}
\newcommand{\rr}{\textbf{r}}
\newcommand{\kk}{\textbf{k}}
\newcommand{\ttt}{\textbf{t}}
\newcommand{\FF}{\textbf{F}}
\newcommand{\XX}{\textbf{X}}
\newcommand{\YY}{\textbf{Y}}
\newcommand{\qqq}{\textbf{q}}
\newcommand{\aaa}{\textbf{a}}
\newcommand{\lep}{\left(}
\newcommand{\rep}{\right)}
\newcommand{\lec}{\left[}
\newcommand{\rec}{\right]}

%\DeclareMathOperator{\sech}{sech}
%\DeclareMathOperator{\csch}{csch}
%\DeclareMathOperator{\arcsec}{arcsec}
%\DeclareMathOperator{\arccot}{arcCot}
%\DeclareMathOperator{\arccsc}{arcCsc}
%\DeclareMathOperator{\arccosh}{arcCosh}
%\DeclareMathOperator{\arcsinh}{arcsinh}
%\DeclareMathOperator{\arctanh}{arctanh}
%\DeclareMathOperator{\arcsech}{arcsech}
%\DeclareMathOperator{\arccsch}{arcCsch}
%\DeclareMathOperator{\arccoth}{arcCoth}

\numberwithin{equation}{section} %sets equation numbers .

%\setcounter{tocdepth}{1} % Show sections
%\setcounter{tocdepth}{2} % + subsections
%\setcounter{tocdepth}{3} % + subsubsections
%\setcounter{tocdepth}{4} % + paragraphs
%\setcounter{tocdepth}{5} % + subparagraphs

%----------------------------------------------------------------------------------------
%	OUTROS PACOTES 
%----------------------------------------------------------------------------------------

\usepackage{cancel} % Cancelar termos em equações
\usepackage{physics} % Pacote muito útil!!!
\usepackage{natbib}
\usepackage{bbm}
\usepackage{enumitem} 
\usepackage{cancel} %cancelar termos em equação
\usepackage{tcolorbox} %quadro de destaque
\usepackage{tikz-cd} %diagrama comutativo
\usepackage{wrapfig}% texto contornando a figura
\usepackage{subcaption} %multiplas figuras
\usepackage{upgreek}
\usepackage[auto-label]{exsheets}
\usepackage{blindtext}  % texto mudo


%----------------------------------------------------------------------------------------
%	DOCUMENTO
%----------------------------------------------------------------------------------------


\begin{document}

%----------------------------------------------------------------------------------------
%	PRÉ-TEXTO
%----------------------------------------------------------------------------------------

%\renewcommand{\maketitlehookb}{\centering You won't expect the results}
\title{Título\\ \vspace{.5cm}}
\subtitle{(sub)\\ \vspace{.5cm} \noindent\rule{15cm}{1.4pt}\\ \vspace{2cm}}
\author{Autor \\ \vspace{13cm}}
\date{ano}

%\pagenumbering{roman} %This command sets the page numbers to lowercase Roman numerals. 

\maketitle
\thispagestyle{empty}
\newpage

\pagenumbering{roman} %This command sets the page numbers to lowercase Roman numerals.

\setlength\parindent{24pt}

\tableofcontents %******************
\newpage





%----------------------------------------------------------------------------------------
%	CAPÍTULOS / SEÇÕES
%----------------------------------------------------------------------------------------
\thispagestyle{empty}
%----------------------------------------------------------------------------------------
%	INTRO
%----------------------------------------------------------------------------------------
\newpage

%**********************************
\section*{Introduç\~ao}
%**********************************

\blindtext  %%Apagar
\blindenumerate  %%Apagar
\blindtext[4]  %%Apagar % Introdução

\setcounter{page}{1} %This will manually set the page counter to 1 in this page, subsequent pages are numbered starting the count from this one. 

\pagenumbering{arabic} %The page numbering is switched to Arabic, this will also restart the page counter. 

\part{Fundamentos}

%----------------------------------------------------------------------------------------
%	CAP 1
%----------------------------------------------------------------------------------------
\newpage

%**********************************
\section{Título}
%**********************************

\blindmathpaper  %%Apagar
\Blindtext  %%Apagar   % capítulos

\part{Aplicações}

%----------------------------------------------------------------------------------------
%	CAP 2
%----------------------------------------------------------------------------------------
\newpage

%**********************************
\section{Título}
%**********************************

\blindmathpaper  %%Apagar
\Blindtext  %%Apagar   % capítulos


\end{document}
